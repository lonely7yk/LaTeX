% !Mode:: "TeX:UTF-8"

\documentclass[12pt,UTF8]{ctexart}
% \documentclass[12pt,type=master]{thuthesis}
\usepackage{amsmath}
\usepackage{bm}%此包与上一个包配合使用, 可用\bm命令同时加粗行中公式与单列公式
\usepackage{amsfonts}
\usepackage{amssymb}
\usepackage{cite}
\usepackage{graphicx}%加载jpg图片
\usepackage[CJKbookmarks]{hyperref}
\usepackage{abstract}
\usepackage{bbm}

%\linespread{1.3}%调整行间距
\usepackage{fancyhdr}%页眉页脚设置
\pagestyle{fancy}%页眉页脚设置

\usepackage{geometry}%页边距
\geometry{left=3.9cm,right=3.9cm,top=3.9cm,bottom=3.9cm}



\begin{document}
% \title{\huge{经典和量子PageRanks的比较}}
% \date{}
% \maketitle
% \thispagestyle{empty}%去掉本页页码

\section*{经典和量子PageRanks的比较}


\begin{abstract}
	摘要根据量子PageRanking的最新发展,我们对基于离散时间和连续时间量子的PageRank算法进行了比较分析。相对于经典PageRank和不同程度,量子测度更好地突出了次要中心,并解决了对于我们研究的所有网络的外围节点间的排名退化。对于离散时间情况,我们研究了广泛网络的步行器概率分布的周期性质,发现主要周期不随网络的大小而增长。在此基础上,我们引入了一种新的量子测量方法,它使用了在前两个周期内相关步行器的最大概率。这一点尤其重要,因为它导致了一种根据网络大小可伸缩的量子PageRanking方案。
\end{abstract}

% \newpage
% \pagenumbering{Roman}
% \tableofcontents
% \newpage
% \listoffigures
% \newpage
% \listoftables
% \newpage
% \pagenumbering{arabic}


\section{介绍}
表征图中节点的相对重要性是网络分析中的关键要素。这种中心性度量的普遍应用是Google的PageRank算法,其中万维网(WWW)被视为由超链接(有向边)将网页(节点)连接的网络。 通过根据其PageRank中心性对每个网页进行排名,搜索引擎的结果将根据其近似质量进行排序。

最近人们对制定PageRank的量子版本很感兴趣。由于Google的PageRank背后的直觉是爬行WWW的经典“随机冲浪者”,因此可以遍历相关定向网络的量子步行器被认为可以提供类似的PageRank度量。 作为经典随机行走的量子模拟,量子行走可以作为量化算法的基本构成,该算法比经典的PageRank算法效果更好。因此,研究它们的量子力学性能是否比谷歌的经典PageRank算法更具优势是很有意思的。

Paparo和Martin-Delgado以及Sánchez-Burillo等人分别提出了两个量子PageRank度量。 前者基于离散时间量子行走(DTQW),而后者使用连续时间量子步(CTQW)。 虽然量子行走已经在任意无向图上被很好地定义了,但由于行走的统一性和可逆性的要求,将这个框架扩展到定向量子行走并不容易。为了解决这个难题,离散时间量子PageRank使用非完全定向但单一的步行,而连续时间算法使用开放系统量子步行而放弃了统一性。

Paparo等人进一步分析了他们在复杂网络上提出的量子PageRank,特别是层次图,有向无标度图和ER随即图。量子PageRank算法不仅能明显地区分三种图,也能在突出次要中心和提高退化方面表现出鲜明的特征。虽然它在无标度网络上显示出更平滑的幂律特性,但它对中心的协调攻击比传统的PageRank算法更敏感。

然而,他们尚未考虑基础离散时间量子步行产生可靠量子PageRank所需的时间步数。我们试图通过调查步行器在网络节点之间的概率幅度的振荡性质来解决这个问题。如果要实现一种基于量子系统的高效PageRank方案,这种考虑至关重要。

引用[5]中基于开放系统量子步行的PageRank将方向性建模为步行器与环境的非统一联系。与离散时间情况类似,开放系统PageRank提升了低连通节点的经典秩退化,同时保留了最中心节点的识别。通过扩展,确定是否对离散量子PageRank的其他特征是否反应在开放系统方案是有用的。

在本文中,我们主要遵循[7]中的分析,但是以三种方式扩展它。首先,我们考虑基于离散时间量子步行的PageRank所涉及的时间尺度。对于此处考虑的网络类型,我们测量了步行器演变的适当数量的时间步长,之后我们可以获得可靠的PageRank。我们提出这样一个上限,不会随着网络规模的增加而显着扩展。其次,我们提出了一个基于每个节点上步行器实现的最大概率幅度的PageRank的替代指标,而不是采用步行者概率分布的时间平均值。这在[8]中作为无向图的中心度量被研究过。第三,我们同时分析了一个基于开放系统的PageRank算法。在我们对三个量子PageRank方案的比较研究中,我们讨论了它们在提取关于考虑下网络的实用信息方面的相对表现。这提供了对作为基于量子步行的复杂网络分析工具的每个方案的更好理解。我们的结果表明,按照经典的PageRank,量子PageRanking在外平面分层、无标度和ER网络系列之间有明显的区别。虽然量子测量选择了更多的次要中心并消除了低节点之间的退化,但每个都表现出不同程度的量子优势。

本文的结构如下:第2节概述了经典和量子PageRank算法的理论框架。在第3节中,我们给出了三种有向网络算法的数值结果,即外平面分层、无标度和ER网络。 我们继续对算法进行比较分析,包括无标度网络上的次要中心分辨率,步行器的本地化-离域化以及无标度网络上的幂律表现。最后,第4节是讨论和结论。

\section{理论}
\subsection{经典PageRank}
Google的PageRank算法是特征向量中心性的变型。PageRank向量 由下式给出

\begin{equation}
	GI_{c1}=I_{c1}
\end{equation}

G是Google矩阵,定义为如下

\begin{equation}
	G:=\alpha E+\frac{1-\alpha}{N}1
\end{equation}

这里$N$是网络阶结点数量,$E$是一个邻接矩阵,$\alpha$是衰减参数(一般$\alpha$为0.85),$1$是一个全为1的矩阵。一般来说,第二项代表步行器跳到网络其他节点的概率。

定义网络邻接矩阵$C$,当结点$k$和$j$有边时$C_{jk}=1$,否则$C_{jk}=0$。为了得到$E$,$C$需要被修改以便每个包含所有0的列$k$(对应于0个出度的节点k1)被一个所有项都设置为$\frac{1}{N}$的列替换。剩下的有出度的列通过除以结点出度正规化到总和为1。将一个节点$k$出度表示为$D_k$,使得$D_k=\sum_{j}C_{jk}$。用数学来表示如下

\begin{equation}
	E_{jk}=\begin{cases}
	\frac{1}{N} & {\rm if}\ D_k=0\\
	\frac{C_{jk}}{D_k} & {\rm if}\ D_k\neq 0
	\end{cases}
\end{equation}

\subsection{离散时间量子行走的Szegedy-Google PageRank}

Szegedy的离散时间的形式是对应于经典随机行走的马尔科夫链的量子化。传统上,对于$N$节点图,这样的过程由转移概率的$N$×$N$矩阵$P$描述,其中$P_jk$表示从节点$k$到节点$j$的转移概率。Szegedy行走在希尔伯特空间$\mathcal { H } ^ { N ^ { 2 } } = \mathcal { H } ^ { N } \otimes \mathcal { H } ^ { N }$发生。这个空间是所有向量$| j , k \rangle$的范围,其中每个向量表示从节点$j$到节点$k$的图中的有向边。

首先,我们定义状态向量如下

\begin{equation}
	\begin{aligned} | \psi _ { j } \rangle & : = | j \rangle \otimes \sum _ { k = 1 } ^ { N } \sqrt { P _ { k j } } | k \rangle \\ & = \sum _ { k = 1 } ^ { N } \sqrt { P _ { k j } } | j , k \rangle \end{aligned}
\end{equation}

对于图的每一个节点$j = 1 , \dots , N$,这表示边状态$| j \rangle _ { 1 } | k \rangle _ { 2 }$的叠加来自第$j$个节点,被$P$赋予权重。投影算子如下

\begin{equation}
	\hat { \Pi } : = \sum _ { j = 1 } ^ { N } | \psi _ { j } \rangle \left\langle \psi _ { j } |\right.
\end{equation}
和
\begin{equation}
	\hat { S } : = \sum _ { j , k = 1 } ^ { N } | j , k \rangle \langle k , j |
\end{equation}

是交换算子。量子行走的步长为统一的算子

\begin{equation}
	\hat { U } : = \hat { S } ( 2 \hat { \Pi } - \hat { \mathbbm { 1 } } )
\end{equation}

然而二步演算算子的形式为

\begin{equation}
	\hat { U } ^ { 2 } : = ( 2 \hat { S } \hat { \Pi } \hat { S } - \hat { \mathbbm { 1 } } ) ( 2 \hat { \Pi } - \hat { \mathbbm { 1 } } )
\end{equation}

正如[4]中提出的,使用Google矩阵$G$作为随机矩阵$P$实现了经典PageRank算法的量子版本。 由于$G$是随机的,因此保持量子行走的单一性; 此外,$G$中保留了有关网络方向性的信息。

相应的量子走路初始化为

\begin{equation}
	| \psi _ { 0 } \rangle = \frac { 1 } { \sqrt { N } } \sum _ { j = 1 } ^ { N } | \psi _ { j } \rangle
\end{equation}

也就是说所有节点之间的相等叠加,但在每个节点的边缘状态之间由$G$加权。以$\hat{U}^2$作为步行的离散时间演化算子,瞬时量子PageRank则是

\begin{equation}
	I _ { q } \left( P _ { i } , t \right) = \left\langle \psi _ { 0 } \left| \hat { U } ^ { \dagger 2 t } \right| i \right\rangle _ { 2 } \left\langle i \left| \hat { U } ^ { 2 t } \right| \psi _ { 0 } \right\rangle
\end{equation}

它被定义为在$t$个时间步之后步行器在网络中$P_i$页面的概率分布。 由于由方程式定义的量子行走算子的统一性和可逆性,该值不会及时收敛到任何静态分布。

因为一个量子PageRank度量必须为每个节点提供一个特殊的排序,Paparo等人将它定义为步行器时间平均概率分布:

\begin{equation}
	I _ { \mathrm { TA } } \left( P _ { i } \right) : = \left\langle I _ { q } \left( P _ { i } , t \right) \right\rangle = \frac { 1 } { t _ { \mathrm { max } } } \sum _ { t = 0 } ^ { t _ { \max } - 1 } I _ { q } \left( P _ { i } , t \right)
\end{equation}

它会在足够大的$t_{max}$时收敛。这在这篇文章中被称为时间平均PageRank度量。

我们根据在节点上找到步行器的峰值概率提出了另一种PageRank度量。我们使用在$t_{max}$之后达到的最大$I_q(P_i,t)$作为节点的量子PageRank:

\begin{equation}
	I _ { P _ { \max } } \left( P _ { i } \right) : = \max \left\{ I _ { q } \left( P _ { i } , t \right) : 1 \leq t \leq t _ { \max } , t \in \mathbbm { Z } \right\}
\end{equation}

我们想基于公式(10)中的$I_q(P_i,t)$的振荡演化来测量一个合适的时间量程$t_{max}$。首先,我们使用$t$=500倍的$\hat{U}^2$步长作为初始状态。对时间序列$I_q(P_i,t)$执行傅立叶变换产生其中存在的振荡频率的功率谱。我们定义为$\omega \left( P _ { i } \right)$噪声之上的最低频率,该噪声使用最高峰值的10%来作为阈值。$I_q(P_i,t)$的周期是$T _ { q } \left( P _ { i } \right) = \frac { 2 \pi } { \omega \left( P _ { i } \right) }$。一般来说,网络中与页面$P_i$关联的每个节点$i$都有一个不同的周期$T_q(P_i)$。

使用$\left\langle T _ { q } ^ { a l l } \right\rangle$来表示所有节点的平均周期。

\begin{equation}
	\left\langle T _ { q } ^ { a l l } \right\rangle : = \frac { 1 } { N } \sum _ { i = 1 } ^ { N } T _ { q } \left( P _ { i } \right)
\end{equation}

\begin{figure}[h]
	\centering
	\includegraphics{part/pic/table1.png}
	\caption{\ 对于$N$个节点,平均周期$\left\langle T _ { q } ^ { 5 } \right\rangle$和三个网络类型的$\left\langle T _ { q } ^ { a l l } \right\rangle$}
\end{figure}

令$\left\langle T _ { q } ^ { 5 } \right\rangle$为瞬时量子PageRank节点的平均周期,其瞬时量子PageRanks的$I _ { q } \left( P _ { j } , t \right)$在各自周期$1 \leq t \leq T _ { q } \left( P _ { j } \right) , t \in \mathbbm { Z }$内达到最高峰值。

按照上述步骤,我们计算与本研究相关的定向网络系列的$\left\langle T _ { q } ^ { a l l } \right\rangle$和$\left\langle T _ { q } ^ { 5 } \right\rangle$,即外平面分层、无标度和ER网络,大小为$N$ = 32,54,128,256,512个节点。我们使用对于每个$N$十个无标度和ER随机网络的集合,这些网络由NetworkX [13]生成。对于此处和本文中的每个ER网络,边连接的概率设置为$p$ = 0.07。 我们使用$t_max = 2 \left\langle T _ { q } ^ { 5 } \right\rangle$作为我们的PageRank分析所需的时间尺度,推断在确定每个网络的一般时间尺度时,大多数中心节点的周期应该比周边节点的周期更强。

数值结果如图1所示,图2绘制了平均周期与网络规模的比例。我们的结果表明,$t_{max} = 2 \left\langle T _ { q } ^ { 5 } \right\rangle$不会随$N$线性向上扩展,而是对于此处考虑的网络类型保持稳定。在确定性构建的外平面分层网络的情况下,对于连续过程中,在大约$\langle T \rangle = 20$时间步长的平均周期平稳。总体而言,无标度网络的平均周期最高。我们看到较大的ER网络(具有相同的边缘概率$p$ = 0.07)倾向于具有较小的平均周期。我们预计较高$N$的时间尺度将保持有限。

这是一个重要的发现,因为它提供了有效实施量子PageRank方案的可能性,这是文献中尚未解决的关键问题。请注意Chiang等人[14]提出了一个有效的量子电路来实现Szegedy在任意稀疏网络上的行走。 尽管在$I_{TA}$和$I_{OS}$中使用Google矩阵会导致相关的统一演化算子$\hat{U}$变得密集,但Loke和Wang[15]利用将Google矩阵划分为可管理的子集并扩展了Chiang等人的方案。只要原始网络稀疏,就可以有效地实现Google矩阵。

\begin{figure}[h]
	\centering
	\includegraphics{part/pic/fig1.png}
	\caption{$N$ = 32,64,128,256,512个节点的网络大小的平均周期缩放。 外平面分层,无标度和ER网络的平均周期$\left\langle T _ { q } ^ { 5 } \right\rangle$(实线)和$\left\langle T _ { q } ^ { all } \right\rangle$(虚线)分别绘制为蓝色,橙色和绿色。对于无标度和ER随机网络,每个N使用十个图的集合。每个误差条对应于来自每个集合的十个$\left\langle T _ { q } \right\rangle$值的平均值的标准误差}
\end{figure}

\subsection{通过连续时间量子遍历开放系统PageRank}

连续时间量子步行最初是由Farhi和Gutmann[16]提出的,它是根据决策树重新计算的计算问题的研究。遵循薛定谔方程,这种演化被描述为

\begin{equation}
	\frac { \mathrm { d } | \Psi ( t ) \rangle } { \mathrm { d } t } = - i \hat { H } | \Psi ( t ) \rangle
\end{equation}

其中$\hat{H}$是转换率矩阵。在量子力学中需要单一演化算子意味着$\hat{H}$必须是Hermitian,通常不是定向行走的情况。为了向CTQW引入方向性,我们采用Lindblad-von Neumann方程的开放系统方法,该方程解释了通过与外部环境配合的有向步行的非单一性质。

为了使用开放量子系统,密度算子的概念被用来替代量子力学中波函数。系统的密度算子由[17]定义

\begin{equation}
	\rho = \sum _ { i = 1 } ^ { N } p _ { i } | \Psi _ { i } \rangle \left\langle \Psi _ { i } |\right.
\end{equation}

其中$p_i$是表示状态多少的常量,$| \Psi _ { i } \rangle$是最终混合状态,其中$\sum _ { i = 1 } ^ { N } p _ { i } = 1$。

Lindblad-von Neumann方程描述了量子系统在追踪环境后如何演变,并且可以以[18]的形式编写:

\begin{equation}
	\frac { \mathrm { d } \rho } { \mathrm { d } t } = - i \hbar [ \hat { H } , \rho ] + \sum _ { k } \gamma _ { k } \left( \hat { L } _ { k } \rho \hat { L } _ { k } ^ { \dagger } - \frac { 1 } { 2 } \left\{ \hat { L } _ { k } \hat { L } _ { k } ^ { \dagger } , \rho \right\} \right)
\end{equation}

其中$\hat{L}_k$是$\rho$所在空间上的单一算子。所有$\hat{L}_k$的集合构成了这个空间的基础。矩阵$\gamma$描述了诸如温度之类的非节能现象如何影响系统。

在经典(无向)和经典(有向)行为之间的顶部参数插值,衰减参数$\beta$被引入到Lindblad-von Neumann方程:

\begin{equation}
	\frac { \mathrm { d } \rho } { \mathrm { d } t } = - i ( 1 - \beta ) [ \hat { H } , \rho ] + \beta \sum _ { i , j } \gamma _ { i j } \left( \hat { L } _ { i j } \rho \hat { L } _ { i j } ^ { \dagger } - \frac { 1 } { 2 } \left\{ \hat { L } _ { i j } \hat { L } _ { i j } ^ { \dagger } , \rho \right\} \right)
\end{equation}

其中$\hat{H}$是邻接矩阵$C$的对称形式,表示其对应的无向图。 根据方程(3),将$\gamma$作为邻接矩阵$E$。 这是当$\alpha=1$时公式(2)中Google矩阵G的特定情况。我们的方法等同于Sánchez-Burillo等人开发的量子PageRank算法,他们设置$\gamma = G$且$\alpha= 0.9$。

我们通过Saalfrank使用的本征算子方法求解主方程。这是一种线性化方法,将非线性方程转化为线性方程,由此使方程(17)变成

\begin{equation}
	\frac { \mathrm { d } \rho } { \mathrm { d } t } = - i ( 1 - \beta ) \mathcal { L } _ { H } \rho + \beta \mathcal { L } _ { D } \rho = \mathcal { L } _ { S O } \rho
\end{equation}

这使得从原始$N$维空间的本征操作符$\mathcal { L } _ { H }$和$\mathcal { L } _ { D }$到$N^2$维的空间,并且在该设置中对密度矩阵进行矢量化。

根据公式(18),$\mathcal { L } _ { H }$和$\mathcal { L } _ { D }$可以组合成一个算子$\mathcal { L } _ { SO }$。对于与时间无关的$\mathcal { L } _ { SO }$,通过取$\mathcal { L } _ { SO }$的矩阵指数,在任何时间$t$都可以很容易地求解这种形式,即

\begin{equation}
	\rho = \rho _ { 0 } e ^ { \mathcal { L } _ { S O t } }
\end{equation}

保证了对于足够大的$t$的稳定结果的收敛,每个节点的占用概率表示开放系统量子PageRank,即

\begin{equation}
	I _ { \mathrm { OS } } \left( P _ { i } \right) : = \langle i | \rho | i \rangle = \rho _ { i i }
\end{equation}
\input{part/result}
% \input{part/part4}\newpage
% \input{part/part5}\newpage
% \input{part/part6}\newpage
% \input{part/part7}\newpage

\end{document}
