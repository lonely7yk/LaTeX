%!TEX program = xelatex

\documentclass{article} 
\usepackage[UTF8]{ctex}
\usepackage{amsmath}	% 这个包必须加

\begin{document} 
	%% 方程
	爱因斯坦的$E=mc^2$方程
	\begin{equation}
		E=mc^2
	\end{equation}

	\[E=mc^2\]
	\[\boxed{E=mc^2}\]

	\[x_{ij}^2\quad \sqrt[2]{x}\]

	\[\pm \times \div \cdot \cap \cup \geq \leq \neq \approx \equiv\]

	% 参数 {ccc} 用于设置每 列的对齐方式,
	% l、c、r 分别表示左中右;\\ 和 & 用来分隔行和列。
	\[\begin{array}{ccc} 
	x_1 & x_2 & \dots \\ 
	x_3 & x_4 & \dots \\ 
	\vdots & \vdots & \ddots \\ 
	\end{array}\]

	% 长公式(不对齐)
	\begin{multline}
		x=a+b+c+\\
		d+e+f+g
	\end{multline}
	% 长公式(对齐)
	\[\begin{split}
		x=&a+b+c+\\
		&d+e+f+g
	\end{split}\]

	% 公式组(不对齐)gather写成gather*去掉编号
	\begin{gather}
		a=b+c+d\\
		x=y+z
	\end{gather}
	% 公式组(对齐)
	\begin{align}
		a&=b+c+d\\
		x&=y+z
	\end{align}
	% case
	\[y=\begin{cases}
	-x & x<0\\
	x & x\ge0
	\end{cases}\]

	% \newtheorem{环境名}[编号延续]{显示名}[编号层次]
	\newtheorem{definition}{定义}[section]
	\newtheorem{theorem}{定理}[section]
	\newtheorem{lemma}[theorem]{引理}
	\newtheorem{corollary}[theorem]{推论}

	\begin{definition}
		Java是一种跨平台语言
	\end{definition}

	\begin{theorem}
		咖啡因会使人兴奋。
	\end{theorem}

	\begin{lemma}
		茶和咖啡因都会使人兴奋。
	\end{lemma}

	\begin{corollary}
		晚上喝咖啡会失眠。
	\end{corollary}

\end{document}