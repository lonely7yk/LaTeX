\section{介绍}
表征图中节点的相对重要性是网络分析中的关键要素。这种中心性度量的普遍应用是Google的PageRank算法,其中万维网(WWW)被视为由超链接(有向边)将网页(节点)连接的网络。 通过根据其PageRank中心性对每个网页进行排名,搜索引擎的结果将根据其近似质量进行排序。

最近人们对制定PageRank的量子版本很感兴趣。由于Google的PageRank背后的直觉是爬行WWW的经典“随机冲浪者”,因此可以遍历相关定向网络的量子步行器被认为可以提供类似的PageRank度量。 作为经典随机行走的量子模拟,量子行走可以作为量化算法的基本构成,该算法比经典的PageRank算法效果更好。因此,研究它们的量子力学性能是否比谷歌的经典PageRank算法更具优势是很有意思的。

Paparo和Martin-Delgado以及Sánchez-Burillo等人分别提出了两个量子PageRank度量。 前者基于离散时间量子行走(DTQW),而后者使用连续时间量子步(CTQW)。 虽然量子行走已经在任意无向图上被很好地定义了,但由于行走的统一性和可逆性的要求,将这个框架扩展到定向量子行走并不容易。为了解决这个难题,离散时间量子PageRank使用非完全定向但单一的步行,而连续时间算法使用开放系统量子步行而放弃了统一性。

Paparo等人进一步分析了他们在复杂网络上提出的量子PageRank,特别是层次图,有向无标度图和ER随即图。量子PageRank算法不仅能明显地区分三种图,也能在突出次要中心和提高退化方面表现出鲜明的特征。虽然它在无标度网络上显示出更平滑的幂律特性,但它对中心的协调攻击比传统的PageRank算法更敏感。

然而,他们尚未考虑基础离散时间量子步行产生可靠量子PageRank所需的时间步数。我们试图通过调查步行器在网络节点之间的概率幅度的振荡性质来解决这个问题。如果要实现一种基于量子系统的高效PageRank方案,这种考虑至关重要。

引用[5]中基于开放系统量子步行的PageRank将方向性建模为步行器与环境的非统一联系。与离散时间情况类似,开放系统PageRank提升了低连通节点的经典秩退化,同时保留了最中心节点的识别。通过扩展,确定是否对离散量子PageRank的其他特征是否反应在开放系统方案是有用的。

在本文中,我们主要遵循[7]中的分析,但是以三种方式扩展它。首先,我们考虑基于离散时间量子步行的PageRank所涉及的时间尺度。对于此处考虑的网络类型,我们测量了步行器演变的适当数量的时间步长,之后我们可以获得可靠的PageRank。我们提出这样一个上限,不会随着网络规模的增加而显着扩展。其次,我们提出了一个基于每个节点上步行器实现的最大概率幅度的PageRank的替代指标,而不是采用步行者概率分布的时间平均值。这在[8]中作为无向图的中心度量被研究过。第三,我们同时分析了一个基于开放系统的PageRank算法。在我们对三个量子PageRank方案的比较研究中,我们讨论了它们在提取关于考虑下网络的实用信息方面的相对表现。这提供了对作为基于量子步行的复杂网络分析工具的每个方案的更好理解。我们的结果表明,按照经典的PageRank,量子PageRanking在外平面分层、无标度和ER网络系列之间有明显的区别。虽然量子测量选择了更多的次要中心并消除了低节点之间的退化,但每个都表现出不同程度的量子优势。

本文的结构如下:第2节概述了经典和量子PageRank算法的理论框架。在第3节中,我们给出了三种有向网络算法的数值结果,即外平面分层、无标度和ER网络。 我们继续对算法进行比较分析,包括无标度网络上的次要中心分辨率,步行器的本地化-离域化以及无标度网络上的幂律表现。最后,第4节是讨论和结论。
